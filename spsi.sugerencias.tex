% archivo:    spsi.sugerencias.tex
% asignatura: Seguridad y Protección de Sistemas Informáticos
\documentclass[
  a4paper,
  spanish,
  12pt,
]{scrartcl}

\linespread{1.1}


%-------------------------------------------------------------------------------
%	PAQUETES
%-------------------------------------------------------------------------------

% Idioma

\usepackage[es-noindentfirst, es-tabla]{babel}

% Citas de texto en línea/bloque

\usepackage[autostyle]{csquotes}

% Matemáticas

\usepackage{amsmath, amsthm, amssymb}
\usepackage{mathtools}
\usepackage{commath}
\usepackage{xfrac}

% Fuentes personalizadas para utilizar con XeLaTeX o LuaLaTeX

\usepackage[no-math]{fontspec}
\setmainfont{Libertinus Serif}
\setsansfont{Libertinus Sans}
\setmonofont{Libertinus Mono}

\usepackage[math-style=TeX]{unicode-math}
\setmathfont{Libertinus Math}


% Configuración de microtype

\defaultfontfeatures{Ligatures=TeX,Numbers=Lining}
\usepackage[activate={true,nocompatibility},final,tracking=true,factor=1100,stretch=10,shrink=10]{microtype}
\SetTracking{encoding={*}, shape=sc}{0}

% Enlaces y colores

\usepackage{hyperref}
\usepackage{xcolor}
\hypersetup{
  colorlinks=true,
  citecolor=,
  linkcolor=,
  urlcolor=blue,
}

% Otros elementos de página

\usepackage{enumitem}
%\setlist[itemize]{leftmargin=*}
%\setlist[enumerate]{leftmargin=*}

\usepackage[labelfont=sc]{caption}

\usepackage{booktabs}
\renewcommand\arraystretch{1.5}

% Tikz

\usepackage{tikz}
\usetikzlibrary{babel}
\usepackage{float}

% Números con círculos
\newcommand*\circled[1]{\tikz[baseline=(char.base)]{
            \node[shape=circle,draw,inner sep=2pt] (char) {#1};}}

% Código

\usepackage{listings}
\lstset{
	basicstyle=\ttfamily,%
	breaklines=true,%
	captionpos=b,                    % sets the caption-position to bottom
  tabsize=2,	                   % sets default tabsize to 2 spaces
  frame=lines,
  numbers=left,
  stepnumber=1,
  aboveskip=12pt,
  showstringspaces=false,
  keywordstyle=\bfseries,
  commentstyle=\itshape,
  columns=flexible,
}
%\renewcommand{\lstlistingname}{Listado}

% ENTORNOS

\newtheoremstyle{ejercicio-style}  % Nombre del estilo
{2\topsep}                                  % Espacio por encima
{1.5\topsep}                                  % Espacio por debajo
{\itshape}                                  % Fuente del cuerpo
{0pt}                                  % Identación
{\scshape}                      % Fuente para la cabecera
{.}                                 % Puntuación tras la cabecera
{5pt plus 1pt minus 1pt}                              % Espacio tras la cabecera
{{\thmname{#1}\thmnumber{ #2}}\thmnote{ (#3)}}  % Especificación de la cabecera

\newtheoremstyle{remark-style}
{-\topsep}                                  % Espacio por encima
{2\topsep}                                  % Espacio por debajo
{}                                  % Fuente del cuerpo
{0pt}                                  % Identación
{\itshape}
{.}
{5pt plus 1pt minus 1pt}                              % Espacio tras la cabecera
{}

% Ejercicios y solución
\theoremstyle{ejercicio-style}
\newtheorem{ejer}{Ejercicio}

\theoremstyle{remark-style}
\newtheorem*{sol}{Solución}


% Márgenes
\usepackage[bottom=3.125cm, top=2.5cm, left=3.5cm, right=3.5cm, marginparwidth=70pt]{geometry}

\usepackage{hyphenat}

%-------------------------------------------------------------------------------
%	CONTENIDO
%-------------------------------------------------------------------------------

\begin{document}

\begin{flushright}
  LibreIM\vspace{.5em}

  \textit{Seguridad y Protección de Sistemas Informáticos}

  Grado en Ingeniería Informática

  \textsc{Universidad de Granada}\vspace{.5em}

  \today\vspace{.5em}
\end{flushright}

\begin{flushleft}
  \scshape\Large Observaciones para el examen.
\end{flushleft}

NOTACIÓN:
$\mathcal{A}$ será un alfabeto (conjunto finito no vacío de símbolos).

\begin{ejer}
  Cifrado de Vigenère.
\end{ejer}

\begin{sol}
  El cifrado de Vigenère consiste en una clave $\alpha\in \exp (\mathcal{A})^*$ y sendas funciones $E_\alpha$ y $D_\alpha$, para cifrar y descifrar respectivamente. Podemos definir $E_\alpha: \exp (\mathcal{A})^* \to \exp (\mathcal{A})^*$ como:
\[ E_\alpha (s) = \langle f^{-1}( (f(s_j) + f(\alpha^{len(s)})_j) \bmod  n)\rangle _j. \]

  Teniendo en cuenta que:
  \begin{itemize}
  \item $f$ es la inyección que asigna a cada letra un entero.
  \item Consideramos $\alpha^{len(s)}$ para asegurar la existencia de una letra en la posición $j$. Es decir, estamos repitiendo la clave hasta alcanzar la longitud de la palabra $s$ (truncando si hace falta). Por ejemplo, si $\alpha$ = HOLA, entonces $\alpha^3=$HOLAHOLAHOLA.
  \item $\langle \rangle _j$ representa que es una palabra.
  \item $n$ es el cardinal del alfabeto empleado.
  \end{itemize}

  De modo análogo se define $D_\alpha: \exp (\mathcal{A})^* \to \exp (\mathcal{A})^*$ como:
 \[D_\alpha (s) = \langle f^{-1}( (f(s_j) - f(\alpha^{len(s)})_j) \bmod  n)\rangle _j.\]

 Veamos que $D_\alpha \circ E_\alpha = 1_{\exp(A)^*}$. Consideramos $s=\langle s_j \rangle _j$ una expresión de $\exp(A)^*$. Entonces:

\begin{align*}
  D_\alpha(E_\alpha(s)) &= D_\alpha(\langle f^{-1}( (f(s_j) + f(\alpha^{len(s)})_j) \bmod  n)\rangle _j)\\
  &= \langle f^{-1}((f(f^{-1}( (f(s_j) + f((\alpha^{len(s)})_j)) \bmod  n)) - f(\alpha^{len(s)})_j) \bmod  n)\rangle _j\\
  &= \langle f^{-1}(( (f(s_j) + f((\alpha^{len(s)})_j) \bmod  n) - f(\alpha^{len(s)})_j) \bmod  n)\rangle _j\\
  &= \langle f^{-1}(f(s_j)\bmod  n)\rangle _j\\
  &= \langle f^{-1}(f(s_j))\rangle _j\\
  &= \langle s_j\rangle _j\\
  &= s.
\end{align*}


  Por último, queda un resultado útil para ver que en realidad ambas funciones son la misma con diferente clave.

  Sea $\alpha$ una clave, entonces definiendo $\alpha' = \langle (-\alpha_j) \bmod  n \rangle _j$ tenemos que $E_{\alpha'} = D_\alpha$.
\end{sol}



\begin{ejer}
  Explicar la transformación SubBytes() que es parte del algoritmo simétrico de cifrado AES.
\end{ejer}

\begin{ejer}
  Limitaciones de los sistemas simétricos de cifrado en la comunicación y cómo la criptografíade clave pública los ha resuelto.
\end{ejer}

\begin{ejer}
  Explicar los fundamentos de la criptografía de clave pública y las líneas fundamentales de lafirma a través de la misma.
\end{ejer}

\begin{ejer}
  Enumerar resumidamente las precauciones más destacables a tomar al generar un cículo decomunicación basado en RSA.
\end{ejer}

\begin{ejer}
  Protocolo de intercambio de llaves según el esquema de Diffie-Hellman y explicación de susupuesta fortaleza.
\end{ejer}

\begin{ejer}
  Explicación del criptosistema de ElGamal.
\end{ejer}

\begin{ejer}
  Explicación del algoritmo de firma estándar (DSA).
\end{ejer}

\begin{ejer}
  Rasgos esenciales de SSH: cifrado, funcionamiento, negociación de cifrado para la sesión yautenticación del acceso del usuario al servidor
\end{ejer}

\end{document}
